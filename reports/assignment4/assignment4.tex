% This is "sig-alternate.tex" V1.9 April 2009
% This file should be compiled with V2.4 of "sig-alternate.cls" April 2009
%
% This has been modified for use in the Principles of Computer System Design @
% DIKU, 2010/2011
\documentclass{sig-alternate}
%
% This gets rid of the copyright box which is not needed for PCSD assignments
%
\makeatletter
\def\@copyrightspace{}
\makeatother
%
% For our purposes page numbers are not so bad
%
\pagenumbering{arabic}

%
% Useful packages
%
\usepackage{url}
\usepackage[english]{babel}
%\usepackage[british]{babel}
\usepackage{hyperref}
%\usepackage{graphicx} % uncomment if you are using graphics

\usepackage{listings}
\lstset{
    keywordstyle=\bfseries\ttfamily\color[rgb]{0,0,1},
    identifierstyle=\ttfamily,
    commentstyle=\color[rgb]{0.133,0.545,0.133},
    stringstyle=\ttfamily\color[rgb]{0.627,0.126,0.941},
    showstringspaces=false,
    basicstyle=\footnotesize,
    numbersep=10pt,
    tabsize=2,
    breaklines=true,
    prebreak = \raisebox{0ex}[0ex][0ex]{\ensuremath{\hookleftarrow}},
    breakatwhitespace=false,
    aboveskip={1.5\baselineskip},
    columns=fixed,
    upquote=true,
    extendedchars=true,
    frame=single,
    captionpos=b
}

\usepackage[margin=11pt,font=footnotesize,labelfont=bf]{caption}

\begin{document}

\title{Datanet: Assignment 4}
\subtitle{Protecting the proxy network}

\numberofauthors{1} % This is individual work right???

\author{
\alignauthor
    Daniel Egeberg\\
    \email{egeberg@diku.dk}
}

\maketitle

\begin{abstract}
    This report outlines the extension of an existing distributed and
    anonymizing proxy network with peer-to-peer encryption. First the
    implementation is outlined followed by a brief discussion of the threat
    scenarios and how these are handled. Finally, the performance implications
    of this is discussed, although experiments were unable to be conducted due
    to problems with the network.
\end{abstract}

\section{Implementation}

The encryption was implemented on top of the sample solution for assignment 2.
This didn't include code for interacting with the peer network, so I ported my
own \verb+proxynet+ module from assignment 3.

A \verb+generate_keys.py+ script is now included to generate a private RSA key
stored in \verb+id_rsa+. This key is used for communication with the tracker
and other encryption-enabled peers.

When registering with the tracker, the corresponding public key is sent
along with a content signature based on the body and a nonce. The tracker
registration is now comprised of multiple steps. Initially, the nonce is empty
and a nonce is returned by the tracker. Then another registration attempt is
made where the nonce is used in the content signature. This proves to the
tracker that the peer is indeed in possession of the correct private key, and
so it is verified and added to the peer list along with its public key.

When communicating with other peers, the proprietary \verb+DATANET+ request
method is used, along with a new response status called 700 Encrypted. The
DATANET envelope contains the actual request encrypted with AES-256. The
session key is RSA encrypted with the next peer's public key retrieved from
the tracker. The other peer then responds with an encrypted response with the
same AES session.

To implement this, I used the PyCrypto library. I also used a few utility
functions from tlslite to compute Base64 encodings of integers.

A peer is currently running on \verb+degeberg.com+ on port 8000.


\section{Threats}

One problem this addresses is that people outside the network cannot see the
data exchange. Only the end points are unencrypted. Another issue is data
integrity. Even though a request may be routed through many places to the
destination, only the few select peers will have access to the plaintext
request and response.


\section{Security improvements}

The above threats are remedied because of the encryption. Seeing as we use
asymmetric encryption, we ensure that only the intended recipient is able to
decrypt the payload.

The tracker also ensures that peers can only register with public keys where
they own the corresponding private key using the multi-step registration
outlined in section~1.


%\section{Security flaws}

\section{Anonymity}

Anonymity is somewhat improved by each peer in the forward chain only knowing
the IP address of the previous and next peer. The destination has no knowledge
of who the originating request came from. The encryption between the peers
ensures that people from the outside of the network cannot ``look into'' the
network to see what is going on. However, it only requires one compromised
peer to compromise that aspect of the anonymity. Of course the larger the
network, the lower the probability of getting selected as a peer.

The data sent through the network is not secured, however. In fact, the
security of that information is decreased because it is exposed to more
people. To fix this issue, the origin would have to encrypt the payload using
the destination's public key so the intermediate peers are unable to eavesdrop
on the communication exchange.


\section{Performance}

I was unable to conduct the performance experiments because I couldn't get
encrypted connections working with the super peers. I decided to entirely
ignore peers that weren't registered as super peers because past experiences
from assignment 3 showed that they were exceedingly unreliable.

When contacting the super peer with an encrypted payload using the DATANET
envelope scheme, I was unable to get no response, while I got the expected
response with an unencrypted request. Other students experienced the same
issues, and because I would expect an error message from the peer instead of
\emph{no} response, I concluded that the super peers were not working properly
at the time I tried testing. I used the \verb+Max-Forwards+ header to make
sure that the super peer in question would be the \emph{only} peer after my
own.

However, the system would have problem with large files because the encryption
scheme between peers doesn't support streaming. Downloading a 100~MB file
would require downloading the entire file before encrypting it and then
sending the entire encrypted payload to the previous peer. This would be
repeated between each pair of peers. This is obviously a problem as it
produces a significant increase in latency, and you would risk peers timing
out.

%\bibliographystyle{abbrv}
%\bibliography{pcsd}  % pcsd.bib is the name of the Bibliography in this case
% You must have a proper ".bib" file
%  and remember to run:
% latex bibtex latex latex
%  or
% pdflatex bibtexx pdflatex pdflatex
% to resolve all references

%APPENDICES are optional
%\balancecolumns
%\appendix
%Appendix A
%\section{Headings in Appendices}

\end{document}
