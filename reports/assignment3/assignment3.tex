% This is "sig-alternate.tex" V1.9 April 2009
% This file should be compiled with V2.4 of "sig-alternate.cls" April 2009
%
% This has been modified for use in the Principles of Computer System Design @
% DIKU, 2010/2011
\documentclass{sig-alternate}
%
% This gets rid of the copyright box which is not needed for PCSD assignments
%
\makeatletter
\def\@copyrightspace{}
\makeatother
%
% For our purposes page numbers are not so bad
%
\pagenumbering{arabic}

%
% Useful packages
%
\usepackage{url}
\usepackage[english]{babel}
%\usepackage[british]{babel}
\usepackage{hyperref}
%\usepackage{graphicx} % uncomment if you are using graphics

\usepackage{listings}
\lstset{
    keywordstyle=\bfseries\ttfamily\color[rgb]{0,0,1},
    identifierstyle=\ttfamily,
    commentstyle=\color[rgb]{0.133,0.545,0.133},
    stringstyle=\ttfamily\color[rgb]{0.627,0.126,0.941},
    showstringspaces=false,
    basicstyle=\footnotesize,
    numbersep=10pt,
    tabsize=2,
    breaklines=true,
    prebreak = \raisebox{0ex}[0ex][0ex]{\ensuremath{\hookleftarrow}},
    breakatwhitespace=false,
    aboveskip={1.5\baselineskip},
    columns=fixed,
    upquote=true,
    extendedchars=true,
    frame=single,
    captionpos=b
}

\usepackage[margin=11pt,font=footnotesize,labelfont=bf]{caption}

\begin{document}

\title{Datanet: Assignment 3}
\subtitle{Anonymizing Distributed Proxy}

\numberofauthors{1} % This is individual work right???

\author{
\alignauthor
    Daniel Egeberg\\
    \email{egeberg@diku.dk}
}

\maketitle

\begin{abstract}
    Bla bla\ldots abstract.
\end{abstract}

\section{Implementation}
\label{sec:implementation}

The distributed proxy network is handled by introducing a new object called
a \verb+ProxyManager+ found in the \verb+proxynet+ module. This object is
responsible for storing data about the network as it is received by the
network's tracker. The main server thread runs this object in a new thread.
When the proxy manager thread starts, it registers with the tracker and
receives the latest peer list and whitelist. It registers with the tracker and
gets updated lists again after \verb+min_wait+ seconds have passed.

An additional configuration flag has been added that contains the domain name
of the tracker. The proxy network is \emph{only} used if this setting is
non-empty on startup.

The server's caching capabilities have been disabled by changing the
\verb+cache+ module so it always reports resources as non-cachable and
not cached. This was done to make sure that any problems with the caching
mechanism would not manifest in the testing of the distributed network.


\section{Security Measures}
\label{sec:security}

In order to restrict which requests the client can make, a whitelist has been
implemented. The whitelist consists of a list of fully qualified domain names
given by the peer tracker. Only request URIs that have a domain name in the
whitelist, or whose domain name is a subdomain of a whitelisted domain is
allowed. All other requests will result in a 403 response code.


\section{Error Handling}
\label{sec:errorhandling}

Seeing as peers are only removed from the tracker list daily, you will often
get a lot of peers that are no longer running or for some other reason are
unresponsive. To deal with this problem, all proxy requests have a fixed
timeout. If the request times out, the peer in question is regarded as a
bad peer and removed from the local peer list. That peer will be taken into
consideration again if it is still in the tracker once the server updates with
the tracker. If the request times out, another peer will not be tried and a
504 Gateway Timeout response is given to the client.

One problem with this approach is that a peer may be removed even though
the problem resides with another peer further ahead in the request chain.
As an example, if our request gets routed through proxies A, B and C and
proxy C happens to be a bad peer, we will be removing proxy A from the list
because it from our point of view seems unresponsive, even though it only
timed out because proxy B timed out, which in turn timed out because of proxy
C. There is no easy way around this problem seeing as we have no knowledge
nor control of what happens when we forward a request to another peer. The
only way to make sure we only remove \emph{actual} bad peers is by using the
\verb+Max-Forwards+ header to make sure that the next peer will be the final
one. However, this goes against the purpose of the network.


\section{Known Limitations}
\label{sec:limitations}


\section{Setup}
\label{sec:setup}

My proxy peer is running as a daemon on my VPS in Linode's London datacenter.
The server is running Gentoo Linux and Python 3.2 compiled directly from the
source tarball from upstream seeing as the latest version in Portage is 3.1,
and the server uses things that were added to the standard library in 3.2.

The server's firewall does not restrict port usage, but the proxy server only
binds to a specific port on all network interfaces.


\section{Experiments}
\label{sec:experiments}

In the time of writing, the network was too unreliable to conduct
any meaningful experiments. A lot of the peers in the tracker were
unresponsive or broken in ome way. The super peers were also unresponsive.
\autoref{lst:working} shows how using the \verb+Max-Forwards+ header to make
sure that my own peer is the last (and thus only) peer in the forward chain
results in a correct response. \autoref{lst:unresponsive} shows how routing
through multiple peers fails due to a timeout.

\lstset{caption={Single forward proxy request.},label=lst:working}
\begin{lstlisting}
HEAD http://www.google.dk/ HTTP/1.1
Host: www.google.dk
Max-Forwards: 1

HTTP/1.1 200 OK
X-XSS-Protection: 1; mode=block
Transfer-Encoding: identity
Set-Cookie: NID=47=iKiNhh71bM4jhvwkxCSFBxiwZF0DmBWQRVrKly6hUPwBkLwAfZ6A4CthofJKtOV4ZD43xZlsIogZqt1dp84GYFl8Awg_Zp9d6gro-DPefs6z0BmV-u4eXp8BP804HGxr; expires=Fri, 02-Dec-2011 22:38:25 GMT; path=/; domain=.google.dk; HttpOnly
Accept-Ranges: none
Expires: -1
Connection: close
Cache-Control: private, max-age=0
Date: Thu, 02 Jun 2011 22:38:41 GMT
Content-Type: text/html; charset=ISO-8859-1
\end{lstlisting}

\lstset{caption={Routing through one peer fails.},label=lst:unresponsive}
\begin{lstlisting}
HEAD http://www.google.dk/ HTTP/1.1
Host: www.google.dk
Max-Forwards: 2

HTTP/1.1 504 Gateway Timeout
Content-Length: 205
Accept-Ranges: none
Server: DanielServer
Connection: close
Date: Thu, 02 Jun 2011 22:40:03 GMT
Content-Type: text/html; charset=utf-8
\end{lstlisting}


%\bibliographystyle{abbrv}
%\bibliography{pcsd}  % pcsd.bib is the name of the Bibliography in this case
% You must have a proper ".bib" file
%  and remember to run:
% latex bibtex latex latex
%  or
% pdflatex bibtexx pdflatex pdflatex
% to resolve all references

%APPENDICES are optional
%\balancecolumns
%\appendix
%Appendix A
%\section{Headings in Appendices}

\end{document}
