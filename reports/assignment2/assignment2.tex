% This is "sig-alternate.tex" V1.9 April 2009
% This file should be compiled with V2.4 of "sig-alternate.cls" April 2009
%
% This has been modified for use in the Principles of Computer System Design @
% DIKU, 2010/2011
\documentclass{sig-alternate}
%
% This gets rid of the copyright box which is not needed for PCSD assignments
%
\makeatletter
\def\@copyrightspace{}
\makeatother
%
% For our purposes page numbers are not so bad
%
\pagenumbering{arabic}

%
% Useful packages
%
\usepackage{url}
\usepackage[english]{babel}
%\usepackage[british]{babel}
\usepackage{hyperref}
%\usepackage{graphicx} % uncomment if you are using graphics

\usepackage{listings}
\lstset{
    keywordstyle=\bfseries\ttfamily\color[rgb]{0,0,1},
    identifierstyle=\ttfamily,
    commentstyle=\color[rgb]{0.133,0.545,0.133},
    stringstyle=\ttfamily\color[rgb]{0.627,0.126,0.941},
    showstringspaces=false,
    basicstyle=\footnotesize,
    numbersep=10pt,
    tabsize=2,
    breaklines=true,
    prebreak = \raisebox{0ex}[0ex][0ex]{\ensuremath{\hookleftarrow}},
    breakatwhitespace=false,
    aboveskip={1.5\baselineskip},
    columns=fixed,
    upquote=true,
    extendedchars=true,
    frame=single,
    captionpos=b
}

\usepackage[margin=11pt,font=footnotesize,labelfont=bf]{caption}

\begin{document}

\title{Datanet: Assignment 2}
\subtitle{HTTP Proxy Server}

\numberofauthors{1} % This is individual work right???

\author{
\alignauthor
    Daniel Egeberg\\
    \email{egeberg@diku.dk}
}

\maketitle

\begin{abstract}
    This report outlines the implementation of HTTP proxy server functionality
    in the previously developed HTTP server. First the implementation is
    described, followed by an account of the caching strategy used, including
    how concurrent connections are handled with respect to caching.

    Then follows a number of experiments measuring the performance of the
    proxy. It is seen that unless the proxy is used to download large files,
    and the files are in cache, there is a speed penalty when using the proxy.

    Finally, the server's limitations are accounted for.
\end{abstract}

\section{Proxy implementation}
\label{sec:implementation}

Most of the proxy implementation has been handled by extending the
\verb+httplib+ module (called \verb+http+ in the previous version). Python's
\verb+urllib.parse+ module is used to parse the request URI. If the
\verb+netloc+ part is non-empty, it is considered a proxy request.

Proxy requests are handled by the new \verb+serve_remote+ function, which
makes the request to the remote server and handles caching. Checking whether
or not something should be cached is delegated to the new \verb+cache+
module (see \autoref{sec:cachestrategy}). The \verb+cache+ module has a
\verb+Manager+ object that is stored in the parent server thread. This object
is shared between the various worker threads and is used to store cache
metadata. The cache itself is stored on the file system. Each cached resource
is given an identifier for use in the metadata cache and on the filesystem.
This is currently implemented as the MD5 hash of the originating URI.


\section{Caching strategy}
\label{sec:cachestrategy}

The server implements support for a subset of the facilities provided
in RFC~2616. The \verb+Expires+ header is fully supported. The
\verb+Cache-Control+ response header is partially supported as it only checks
the \verb+private+, \verb+no-cache+, \verb+no-store+ and \verb+max-age+
values. If either of the first three values are present, the resource will
not be cached. If the last value is present, the resource will be cached and
considered valid for the specified duration. The remaining possible values
are ignored, as is the \verb+Cache-Control+ header in a request context. The
\verb+Cache-Control+ header takes higher precedence than the \verb+Expires+
header.

When a client requests a resource that is valid in the proxy's cache the
\verb+If-Modified-Since+ and \verb+If-None-Match+ request headers are used to
check if the client should receive a 304 Not Modified or 200 OK response code.
In the case of a 200 OK, the resource is loaded from the disk in the proxy's
cache directory.

\subsection{Concurrency}
\label{sec:concurrency}

Concurrent requests are handled by \emph{always} downloading a resource from
the remote location unless the resource is marked as valid in the proxy's
cache. When downloading the resource, it is downloaded into a temporary file
in the cache directory with a random number. Only when the file is entirely
downloaded will it be stored in the cache, and the file is moved to its proper
location based on the generated resource id.

This strategy means that if two clients request the same uncached resource
simultaneously, the copy stored in the cache will be the one that has most
recently finished downloading. This, however, means that it will be downloaded
multiple times and bandwidth is thus ``wasted''.


\section{Experiments}
\label{sec:experiments}

To test the performance, we setup two test scenarios and measure the response
time with and without the proxy.

\subsection{Websites}
\label{sec:websites}

In the first test we use a browser to make requests to
\url{http://slashdot.org} and \url{http://www.reddit.com}, which both contain
references to many resources such as images, CSS, Javascript, etc. We test
both with the browser's cache and the proxy's cache, and both with a primed
and unprimed cache. \autoref{tbl:websites} shows the results of this.

\begin{table}[h]
    \centering
    \begin{tabular}{|l|l|l||r|}
        \hline
        \bf{Website} & \bf{Cache} & \bf{Cached} & \bf{Response time} \\
        \hline
        \hline
        \texttt{slashdot.org} & Browser & No & 4.18s \\ \hline
        \texttt{slashdot.org} & Browser & Yes & 2.54s \\ \hline
        \texttt{slashdot.org} & Proxy & No & 5.00s \\ \hline
        \texttt{slashdot.org} & Proxy & Yes & 3.31s \\ \hline \hline
        \texttt{reddit.com} & Browser & No & 2.24s \\ \hline
        \texttt{reddit.com} & Browser & Yes & 1.34s \\ \hline
        \texttt{reddit.com} & Proxy & No & 5.61s \\ \hline
        \texttt{reddit.com} & Proxy & Yes & 3.36s \\ \hline
    \end{tabular}
    \caption{Response times for reddit and Slashdot.}
    \label{tbl:websites}
\end{table}

\subsection{Large file}
\label{sec:largefile}

In this test we download the Ubuntu x86 desktop install image (693 MB).
\autoref{tbl:largefile} shows the speed and time to download the file using a
direct connection, using the proxy and using the proxy with the file in the
local cache.

\begin{table}[h]
    \centering
    \begin{tabular}{|l||r|r|}
        \hline
        \bf{Type} & \bf{Download speed} & \bf{Time} \\
        \hline
        \hline
        Direct & 3.96 MB/s & 2:55.12 \\ \hline
        Proxy & 2.69 MB/s & 4:17.77 \\ \hline
        Proxy (cached) & 113 MB/s & 6.14 \\ \hline
    \end{tabular}
    \caption{Download speed and time for a 693 MB file.}
    \label{tbl:largefile}
\end{table}

\subsection{Conclusion}
\label{sec:testconclusions}

The tests in \autoref{sec:websites} and \autoref{sec:largefile} show that we
incur a speed penalty when using the proxy instead of a direct connection. The
response time for the websites was higher, and the large file was downloaded
at a slower average speed when using the proxy. However, when using the proxy,
there is in all cases a performance gain when the requested resource is in the
proxy's cache.

The large file was---unsurprisingly---retrieved much faster when it was
already in the proxy cache. Thus, for this proxy server there is only a
speed advantage for the \emph{end user} when the requested resource is large
and there is a significant disparity between the download speed from the
resource's location and the download speed from the proxy's location.


\section{Known limitations and bugs}
\label{sec:bugs}

In addition to the limitations outlined in the first report, the
implementation has the following known limitations and bugs:

\begin{itemize}
    \item The only supported request methods are GET and HEAD.
    \item There is no support for HTTPS.
    \item The proxy server will \emph{always} override the \verb+Transfer-Encoding+ to \verb+identity+.
    \item The cache on the disk is never cleaned up. Neither when shutting down, or when a resource on runtime is declared expired.
    \item The cache is not persistent between server restarts.
    \item Only a subset of the caching mechanisms outlined in RFC~2616 is supported (see \autoref{sec:cachestrategy}).
    \item There is a performance loss when using the proxy (see \autoref{sec:experiments}).
    \item The \verb+Vary+ header is not honered.
\end{itemize}

%\bibliographystyle{abbrv}
%\bibliography{pcsd}  % pcsd.bib is the name of the Bibliography in this case
% You must have a proper ".bib" file
%  and remember to run:
% latex bibtex latex latex
%  or
% pdflatex bibtexx pdflatex pdflatex
% to resolve all references

%APPENDICES are optional
%\balancecolumns
%\appendix
%Appendix A
%\section{Headings in Appendices}

\end{document}
