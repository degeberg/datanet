% This is "sig-alternate.tex" V1.9 April 2009
% This file should be compiled with V2.4 of "sig-alternate.cls" April 2009
%
% This has been modified for use in the Principles of Computer System Design @
% DIKU, 2010/2011
\documentclass{../sig-alternate}
%
% This gets rid of the copyright box which is not needed for PCSD assignments
%
\makeatletter
\def\@copyrightspace{}
\makeatother
%
% For our purposes page numbers are not so bad
%
\pagenumbering{arabic}

%
% Useful packages
%
\usepackage{url}
\usepackage[english]{babel}
%\usepackage[british]{babel}
\usepackage{hyperref}
%\usepackage{graphicx} % uncomment if you are using graphics

\usepackage{lipsum}

\begin{document}

\title{Datanet: Assignment 1}

\numberofauthors{1} % This is individual work right???

\author{
\alignauthor
    Daniel Egeberg\\
    \email{egeberg@diku.dk}
}

\maketitle

\begin{abstract}
    \lipsum[1]
\end{abstract}

\section{Web server design}

The web server has been implemented using Python 3.2. It will not run on
earlier versions of Python 3.x due to the usage of the \verb+argparse+
module. No libraries outside of the standard libraries provided by Python has
been used in the implementation. No operating system specific functionality
has been used, so it will likely run on all platforms capable of running
Python 3.2, but it has only been tested on GNU/Linux.

The web server consists of the following components:

\begin{description}
    \item[\texttt{main.py}] is the main entry point for running the web
        server. It parses the arguments and starts the server, optionally as a
        daemon.
    \item[\texttt{Server}] is a class that creates a listening socket
        and starts a pool of \texttt{Worker} threads. It also receives
        incoming connections and places the clients in a client queue. The
        size of the worker pool is 5 by default. These workers are reused
        between connections to prevent the additional overhead from
        continuously creating and killing new threads. If there is no
        available worker for a client, it will be queued.
    \item[\texttt{Worker}] is a class handles the individual client requests.
        It gets the client sockets from the client queue that the
        \texttt{Server} populates. It uses the functionality in the
        \texttt{http} module to send an appropriate response to the client.
    \item[\texttt{http}] is the module that contains the various logic for
        parsing an HTTP request and generating HTTP responses.
    \item[\texttt{template}] is a small module that supports rudimentary
        templating support. This is used for things like error pages and
        directory listings.
\end{description}


\section{Known limitations}

The web server has a number of intentional limitations. The only supported
requests methods are GET and HEAD. POST is supported by the web server, but
is treated as identical to GET requests. All other request methods will
result in a 501 Not Implemented response status code.

The only supported HTTP version is 1.1. HTTP/1.0 requests will be met with
a 505 HTTP Version Not Supported response code. This was chosen so it would
not be necessary dealing with any discrepancies between versions 1.0 and 1.1
there might exist.

Persistens connections are not supported. This is indicated to clients by
always returning a \verb+Connection: close+ header in the response.

The accuracy of the \verb+Content-Type+ header depends on \verb+mimetypes+
module in Python. This module's accuracy depends on the availability of
suitable mapping files on the underlying system. If the \verb+mimetypes+
module is unable to deduce a MIME type based on the file extension, no
\verb+Content-Type+ header is returned. This was not seen as any major issue.

If a worker thread crashes for whatever reason, it will not restart and a
new worker will not be started. This means that if all workers crash,
the web server process will still be running and accepting connections,
but these will simply not be handled.


\section{Test results}
\lipsum[10-14]


\section{Supported headers}
\lipsum[23-24]


\section{Extensibility}
\lipsum


%\bibliographystyle{abbrv}
%\bibliography{pcsd}  % pcsd.bib is the name of the Bibliography in this case
% You must have a proper ".bib" file
%  and remember to run:
% latex bibtex latex latex
%  or
% pdflatex bibtexx pdflatex pdflatex
% to resolve all references

%APPENDICES are optional
%\balancecolumns
%\appendix
%Appendix A
%\section{Headings in Appendices}

\end{document}
