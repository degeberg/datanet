% This is "sig-alternate.tex" V1.9 April 2009
% This file should be compiled with V2.4 of "sig-alternate.cls" April 2009
%
% This has been modified for use in the Principles of Computer System Design @
% DIKU, 2010/2011
\documentclass{sig-alternate}
%
% This gets rid of the copyright box which is not needed for PCSD assignments
%
\makeatletter
\def\@copyrightspace{}
\makeatother
%
% For our purposes page numbers are not so bad
%
\pagenumbering{arabic}

%
% Useful packages
%
\usepackage{url}
\usepackage[english]{babel}
%\usepackage[british]{babel}
\usepackage{hyperref}
%\usepackage{graphicx} % uncomment if you are using graphics

\usepackage{listings}
\lstset{
    keywordstyle=\bfseries\ttfamily\color[rgb]{0,0,1},
    identifierstyle=\ttfamily,
    commentstyle=\color[rgb]{0.133,0.545,0.133},
    stringstyle=\ttfamily\color[rgb]{0.627,0.126,0.941},
    showstringspaces=false,
    basicstyle=\footnotesize,
    numbersep=10pt,
    tabsize=2,
    breaklines=true,
    prebreak = \raisebox{0ex}[0ex][0ex]{\ensuremath{\hookleftarrow}},
    breakatwhitespace=false,
    aboveskip={1.5\baselineskip},
    columns=fixed,
    upquote=true,
    extendedchars=true,
    frame=single,
    captionpos=b
}

\usepackage[margin=11pt,font=footnotesize,labelfont=bf]{caption}

\begin{document}

\title{Datanet: Assignment 1}
\subtitle{Web server}

\numberofauthors{1} % This is individual work right???

\author{
\alignauthor
    Daniel Egeberg\\
    \email{egeberg@diku.dk}
}

\maketitle

\begin{abstract}
    This report outlines various design decisions in the implementation of
    the web server, including known limitations with respect to RFC 2616
    compliance and a list of the headers that are supported.

    Additionally, the testing methods and results are accounted for as well as
    results of benchmarking using Apache Bench.

    Finally, it is described how the implementation will be able to
    accommodate extension to support future assignments' required features.
\end{abstract}

\section{Web server design}

The web server has been implemented using Python 3.2. It will not run on
earlier versions of Python 3.x. No libraries outside of the standard libraries
provided by Python has been used in the implementation. No operating system
specific functionality has been used, so it will likely run on all platforms
capable of running Python 3.2, but it has only been tested on GNU/Linux.

The web server consists of the following components:

\begin{description}
    \item[\texttt{main.py}] is the main entry point for running the web
        server. It parses the arguments and starts the server, optionally as a
        daemon.
    \item[\texttt{Server}] is a class that creates a listening socket
        and starts a pool of \texttt{Worker} threads. It also receives
        incoming connections and places the clients in a client queue. The
        size of the worker pool is 20 by default. These workers are reused
        between connections to prevent the additional overhead from
        continuously creating and killing new threads. If there is no
        available worker for a client, it will be queued.
    \item[\texttt{Worker}] is a class handles the individual client requests.
        It gets the client sockets from the client queue that the
        \texttt{Server} populates. It uses the functionality in the
        \texttt{http} module to send an appropriate response to the client.
    \item[\texttt{http}] is the module that contains the various logic for
        parsing an HTTP request and generating HTTP responses. The
        \texttt{Response} object has various utility functions for generating
        different kinds of responses (errors, serving files, directory
        listings, etc.).
    \item[\texttt{template}] is a small module that supports rudimentary
        templating support. This is used for things like error pages and
        directory listings.
\end{description}

The web server has various settings that can be found in the configuration
file (\verb+config.ini+ in the current directory by default, but can be set
using the \verb+--config+ option). It will bind to 0.0.0.0:80 by default, so
starting it requires superuser privileges unless the port is changed using the
\verb+--port+ option or in the configuration file.


\section{Known limitations}
\label{sec:limitations}

The web server has a number of intentional limitations. The only supported
requests methods are GET and HEAD. POST is supported by the web server, but
is treated as identical to GET requests. All other request methods will
result in a 501 Not Implemented response status code.

The only supported HTTP version is 1.1. HTTP/1.0 requests will be met with
a 505 HTTP Version Not Supported response code. This was chosen so it would
not be necessary dealing with any discrepancies between versions 1.0 and 1.1
there might exist.

Persistent connections are not supported. This is indicated to clients
by always returning a \verb+Connection: close+ header in the response.
Likewise are other ``advanced'' features such as chunked transfers and ranges
unsupported as well.

The accuracy of the \verb+Content-Type+ header depends on the \verb+mimetypes+
module in Python. This module's accuracy depends on the availability of
suitable mapping files on the underlying system. If the \verb+mimetypes+
module is unable to deduce a MIME type based on the file extension, no
\verb+Content-Type+ header is returned. This was not seen as any major issue.

If a worker thread crashes for whatever reason, it will not restart and a
new worker will not be started. This means that if all workers crash,
the web server process will still be running and accepting connections,
but these will simply not be handled.

The server's support for compressing the response entity body is limited to
\verb+deflate+, \verb+gzip+ and \verb+identity+, but could be extended to
support other compression methods as well if necessary.


\section{Test results}

The web server has been tested using \verb+telnet+ and using a web browser
(specifically Firefox 4.0).

\autoref{lst:regular200} shows an HTTP request of the servers configuration
file, \verb+config.ini+ and the generated response. \autoref{lst:etag} shows a
request for the same file, but uses the \verb+If-None-Match+ header and the
file's ETag.

\lstset{caption={Serving a regular file.},label=lst:regular200}
\begin{lstlisting}
GET /config.ini HTTP/1.1
Host: localhost

HTTP/1.1 200 OK
Content-Length: 207
Accept-Ranges: none
Server: DanielServer
Last-Modified: Wed, 27 Apr 2011 08:20:49 GMT
Connection: close
ETag: fc42903e6b856f30aaf6fbac80def0c1
Date: Wed, 27 Apr 2011 10:34:46 GMT
Content-Type: text/plain

[server]
bind = 0.0.0.0
port = 80
webroot = .
read_bufsize = 1024
compression_limit = 1048576
listen_backlog = 5

[resources]
templates = ./templates/

[logs]
error = /tmp/error.log
server = /tmp/server.log
\end{lstlisting}

\lstset{caption={The ETag for \texttt{config.ini} matches, so a 304 Not Modified response is returned instead of the file.},label=lst:etag}
\begin{lstlisting}
GET /config.ini HTTP/1.1
Host: localhost
If-None-Match: fc42903e6b856f30aaf6fbac80def0c1

HTTP/1.1 304 Not Modified
Accept-Ranges: none
Server: DanielServer
Last-Modified: Wed, 27 Apr 2011 08:20:49 GMT
Connection: close
ETag: fc42903e6b856f30aaf6fbac80def0c1
Date: Wed, 27 Apr 2011 10:35:15 GMT
Content-Type: text/plain
\end{lstlisting}

\autoref{lst:notimplemented} shows how the server handles an unsupported HTTP
DELETE request by returning a 501 Not Implemented response code. Additionally,
it shows the templating system. All error pages use the same template as seen
in \autoref{lst:notfound} where a non-existing resource is requested and a 404
Not Found response code is returned to the client.

\lstset{caption={Because HTTP DELETE is not supported, the server returns a 501 Not Implemented response code.},label=lst:notimplemented}
\begin{lstlisting}
DELETE /config.txt HTTP/1.1
Host: localhost

HTTP/1.1 501 Not Implemented
Content-Length: 205
Accept-Ranges: none
Server: DanielServer
Connection: close
Date: Wed, 27 Apr 2011 10:36:57 GMT
Content-Type: text/html; charset=utf-8

<!DOCTYPE HTML>
<html>
<head>
    <meta http-equiv="content-type" content="text/html; charset=utf-8">
    <title>501 Not Implemented</title>
</head>
<body>
    <h1>501 Not Implemented</h1>
</body>
</html>
\end{lstlisting}

\lstset{caption={A correct 404 Not Found response code is returned if the requested resource does not exist.},label=lst:notfound}
\begin{lstlisting}
GET /does_not_exist.txt HTTP/1.1
Host: localhost

HTTP/1.1 404 Not Found
Content-Length: 193
Accept-Ranges: none
Server: DanielServer
Connection: close
Date: Wed, 27 Apr 2011 10:38:08 GMT
Content-Type: text/html; charset=utf-8

<!DOCTYPE HTML>
<html>
<head>
    <meta http-equiv="content-type" content="text/html; charset=utf-8">
    <title>404 Not Found</title>
</head>
<body>
    <h1>404 Not Found</h1>
</body>
</html>
\end{lstlisting}

When a client requests content in encodings using the \verb+Accept-Encoding+
header that the server is unable to provide, a 406 Not Acceptable response
code is returned as in \autoref{lst:notacceptable}. If the server is able to
provide the content in one of the, by the client, acceptable encodings, it
will do as in \autoref{lst:gzip} where gzipped content is returned.

\lstset{caption={A 406 Not Acceptable response code is returned because the server is unable to accomadate the requirements in the \texttt{Accept-Encoding} request header.},label=lst:notacceptable}
\begin{lstlisting}
HTTP/1.1 406 Not Acceptable
Content-Length: 203
Accept-Ranges: none
Server: DanielServer
Connection: close
Date: Wed, 27 Apr 2011 10:57:40 GMT
Content-Type: text/html; charset=utf-8

<!DOCTYPE HTML>
<html>
<head>
    <meta http-equiv="content-type" content="text/html; charset=utf-8">
    <title>406 Not Acceptable</title>
</head>
<body>
    <h1>406 Not Acceptable</h1>
</body>
</html>
\end{lstlisting}

\lstset{caption={The server returns the compressed resource. Entity body is not shown.},label=lst:gzip}
\begin{lstlisting}
GET /http.py HTTP/1.1
Host: localhost
Accept-Encoding: gzip

HTTP/1.1 200 OK
Content-Length: 2790
Content-Encoding: gzip
Accept-Ranges: none
Server: DanielServer
Last-Modified: Wed, 27 Apr 2011 09:56:42 GMT
Connection: close
ETag: 9ca9ba76558be7744d4ff3aed141dbe6
Date: Wed, 27 Apr 2011 10:40:04 GMT
Content-Type: text/x-python

**gzip compressed data**
\end{lstlisting}

Additionally, the server has been benchmarked using Apache
Bench\footnote{\url{http://httpd.apache.org/docs/2.2/programs/ab.html}}.
Benchmarking it by making 1000 requests to a 2 kB file with a concurrency
level of 100 requests yielded a median response time of 7 ms, and the server
was able to handle 332.06 requests per second. For some reason, the last
few requests took 350--400 ms. I was unable to identify the source of this
problem.

For comparison, running the same benchmark on the same file and computer, but
using nginx\footnote{\url{http://wiki.nginx.org/}} (a web server designed
specifically for high-performance) gave a median response time of 8 ms,
serving 10025.16 requests per minute with the longest response taking 17 ms.
My own web server was able to deliver the file that fast in 80\% of the cases.
Given the circumstances and the purpose of the implementation, I think it
performs reasonably well.


\section{Supported headers}

The web server supports the following headers, subject to the limitations
outlined in \autoref{sec:limitations}:

\begin{itemize}
    \item Response headers:
    \begin{itemize}
        \item Server
        \item Connection
        \item Date
        \item Accept-Ranges
        \item Content-Type
        \item Location
        \item Last-Modified
        \item ETag
        \item Content-Encoding
        \item Content-Length
    \end{itemize}
    \item Request headers:
    \begin{itemize}
        \item If-None-Match
        \item If-Modified-Since
        \item Accept-Encoding
    \end{itemize}
\end{itemize}


\section{Extensibility}

Extending the web server to support additional features should not be a
problem. Handling additional request methods can be done by applying the
\verb+register_method_handler+ decorator in the \verb+http+ module. However,
in order to support requests methods that require en entity body (e.g. POST),
it is necessary to extend the \verb+Worker+ class so it also retrieves the
body.

All headers that are received by the client (whether they are recognized or
not) are stored in a dictionary called \verb+req+ that the appropriate request
handler gets. This means that the request handler can use the received headers
it needs and act on them accordingly.

These two design decisions combined should make it easier implementing
features that are required in future assignments.

%\bibliographystyle{abbrv}
%\bibliography{pcsd}  % pcsd.bib is the name of the Bibliography in this case
% You must have a proper ".bib" file
%  and remember to run:
% latex bibtex latex latex
%  or
% pdflatex bibtexx pdflatex pdflatex
% to resolve all references

%APPENDICES are optional
%\balancecolumns
%\appendix
%Appendix A
%\section{Headings in Appendices}

\end{document}
